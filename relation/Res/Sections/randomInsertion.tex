\section{Random Insertion Algorithm}\label{randominsertion}
Idea of the algorithm is to take randomly a node and find the cheapest path going through that random k from pair of nodes in our partial circuit. 
As it was mentiond erlier, the graph initialization functions create graph as a list of edges and also provide us a list of all nodes.
The Implementation of the code follow the based concept:

\begin{itemize}
    \item \textbf{Initialization}: Take first node (we always took first node from the graph and make it as partial circuit;
    \item \textbf{Selection}: Random select node not in partial circuit;
    \item \textbf{Insertion}: Find edge in partial circuit that minimize the triangle inequality: w(i,k)+w(k,j)-w(i,j) and insert k between i and j;
    \item \textbf{Repeat}: step \textbf{Selection} and \textbf{Insertion} until add all nodes to partial circuit.
\end{itemize}


\subsection{Functions}
We implement 3 functions: 

\begin{itemize}
	\item \textbf{RandomInsertion()}: Main function that call other in process of counting the triangle inequality and check the distance.
	\item \textbf{buildPath(p,k)}: Function to get the distance required to get to random k and also position after we need to insert. 
	\item \textbf{minDist(u)}: Function to get closest node to input one.
\end{itemize}

\subsection{Implementation}
The final solution is the following:
\begin{itemize}
	\item We start the time;
	\item We execute a deep copy unvisited of the original list v;
	\item We follow the algorithm and we start from the node 0 \textit{(startingNode)} of the unvisited list;
\end{itemize}

Originality, that we found pretty usefull in minDist function is to run in cicle of data instead of all graph.




\pagebreak