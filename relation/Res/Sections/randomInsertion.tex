\section{Random Insertion Algorithm}\label{randominsertion}
The idea of the random insertion algorithm is to randomly select a node and find the cheapest path going through that random k node from pair of nodes in our partial circuit. 
As it was mentiond erlier, the graph initialization functions create graph as a list of edges and also provide us a list of all nodes.
The Implementation of the code follow the based concept:

\begin{itemize}
    \item \textbf{Initialization}: Take first node (we always took first node from the graph) and make it as partial circuit;
    \item \textbf{Selection}: Random select node not in partial circuit;
    \item \textbf{Insertion}: Find edge in partial circuit that minimizes the triangle inequality: w(i,k)+w(k,j)-w(i,j) and insert k between i and j;
    \item \textbf{Repeat}: step \textbf{Selection} and \textbf{Insertion} until add all nodes to partial circuit.
\end{itemize}


\subsection{Functions}
We implemented 3 functions: 

\begin{itemize}
	\item \textbf{RandomInsertion()}: Main function that call other in process of counting the triangle inequality and check the distance.
	\item \textbf{buildPath(p,k)}: Function to get the distance required to get to random k and also position after we need to insert. 
	\item \textbf{minDist(u)}: Function to get closest node to input one.
\end{itemize}

\subsection{Implementation}
The final solution is the following:
\begin{itemize}
	\item We start the time;
	\item We execute a deep copy unvisited of the original list v;
	\item We follow the algorithm and we start from the node 0 \textit{(startingNode)} of the unvisited list;
	\item We remove the starting node from the unvisited list and we add always the starting node in our path \textit{(is a list)};
	\item We found the vertex j that minimize w(0,j) with the function \textit{minDist} and we remove it from the unvisited list and add it to the path;
	\item We add starting node to the end because it should be a circle list;
	\item We initialize our solution \textit{(dist)} to the distance found with the function \textit{minDist} counted twice and we start our cycle until we have visited all vertices \textit{(len(unvisited)>0))};
	\item In the while cycle:
	\begin{enumerate}
		\item We take the random node from the unvisited list;
		\item We call the function \textit{buildPath}:
		\begin{itemize}
			\item  It initalize the distance to infinite;
			\item  In the cycle it checks for all elements of our path what is the position where the random node k minimize the total weight \textit{"min(w(i,k)+w(k,j)-w(i,j))"};
			\item It returns the position where we have to insert the node and the distance required to add the random node.
		\end{itemize}
		\item We remove the random node from the unvisited list and we add it in the path;
		\item We update the distance and we add the random node in the correct position. 
	\end{enumerate}
	\item When we finish the while cycle we end the time and we calculate the time cost;
	\item We return the solution and the time.\\
\end{itemize}


Regarding originality, we implemented a method in the minDist function that runs through a slice of the graph rather than through the entire graph, which ultimately makes the function more efficient.




\pagebreak
